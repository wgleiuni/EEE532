\documentclass[journal]{IEEEtran}
\usepackage{epsfig}
\usepackage{bm}
\usepackage{amssymb}
\usepackage{amsmath}
\usepackage{graphicx}
\usepackage{epstopdf}
\ifCLASSINFOpdf
% \usepackage[pdftex]{graphicx}
  % declare the path(s) where your graphic files are
  % \graphicspath{{../pdf/}{../jpeg/}}
  % and their extensions so you won't have to specify these with
  % every instance of \includegraphics
  % \DeclareGraphicsExtensions{.pdf,.jpeg,.png}
\else
  % or other class option (dvipsone, dvipdf, if not using dvips). graphicx
  % will default to the driver specified in the system graphics.cfg if no
  % driver is specified.
  % \usepackage[dvips]{graphicx}
  % declare the path(s) where your graphic files are
  % \graphicspath{{../eps/}}
  % and their extensions so you won't have to specify these with
  % every instance of \includegraphics
  % \DeclareGraphicsExtensions{.eps}
\fi
% graphicx was written by David Carlisle and Sebastian Rahtz. It is
% required if you want graphics, photos, etc. graphicx.sty is already
% installed on most LaTeX systems. The latest version and documentation can
% be obtained at: 
% http://www.ctan.org/tex-archive/macros/latex/required/graphics/
% Another good source of documentation is "Using Imported Graphics in
% LaTeX2e" by Keith Reckdahl which can be found as epslatex.ps or
% epslatex.pdf at: http://www.ctan.org/tex-archive/info/
%
% latex, and pdflatex in dvi mode, support graphics in encapsulated
% postscript (.eps) format. pdflatex in pdf mode supports graphics
% in .pdf, .jpeg, .png and .mps (metapost) formats. Users should ensure
% that all non-photo figures use a vector format (.eps, .pdf, .mps) and
% not a bitmapped formats (.jpeg, .png). IEEE frowns on bitmapped formats
% which can result in "jaggedy"/blurry rendering of lines and letters as
% well as large increases in file sizes.
%
% You can find documentation about the pdfTeX application at:
% http://www.tug.org/applications/pdftex





% *** MATH PACKAGES ***
%
%\usepackage[cmex10]{amsmath}
% A popular package from the American Mathematical Society that provides
% many useful and powerful commands for dealing with mathematics. If using
% it, be sure to load this package with the cmex10 option to ensure that
% only type 1 fonts will utilized at all point sizes. Without this option,
% it is possible that some math symbols, particularly those within
% footnotes, will be rendered in bitmap form which will result in a
% document that can not be IEEE Xplore compliant!
%
% Also, note that the amsmath package sets \interdisplaylinepenalty to 10000
% thus preventing page breaks from occurring within multiline equations. Use:
%\interdisplaylinepenalty=2500
% after loading amsmath to restore such page breaks as IEEEtran.cls normally
% does. amsmath.sty is already installed on most LaTeX systems. The latest
% version and documentation can be obtained at:
% http://www.ctan.org/tex-archive/macros/latex/required/amslatex/math/





% *** SPECIALIZED LIST PACKAGES ***
%
%\usepackage{algorithmic}
% algorithmic.sty was written by Peter Williams and Rogerio Brito.
% This package provides an algorithmic environment fo describing algorithms.
% You can use the algorithmic environment in-text or within a figure
% environment to provide for a floating algorithm. Do NOT use the algorithm
% floating environment provided by algorithm.sty (by the same authors) or
% algorithm2e.sty (by Christophe Fiorio) as IEEE does not use dedicated
% algorithm float types and packages that provide these will not provide
% correct IEEE style captions. The latest version and documentation of
% algorithmic.sty can be obtained at:
% http://www.ctan.org/tex-archive/macros/latex/contrib/algorithms/
% There is also a support site at:
% http://algorithms.berlios.de/index.html
% Also of interest may be the (relatively newer and more customizable)
% algorithmicx.sty package by Szasz Janos:
% http://www.ctan.org/tex-archive/macros/latex/contrib/algorithmicx/




% *** ALIGNMENT PACKAGES ***
%
%\usepackage{array}
% Frank Mittelbach's and David Carlisle's array.sty patches and improves
% the standard LaTeX2e array and tabular environments to provide better
% appearance and additional user controls. As the default LaTeX2e table
% generation code is lacking to the point of almost being broken with
% respect to the quality of the end results, all users are strongly
% advised to use an enhanced (at the very least that provided by array.sty)
% set of table tools. array.sty is already installed on most systems. The
% latest version and documentation can be obtained at:
% http://www.ctan.org/tex-archive/macros/latex/required/tools/


%\usepackage{mdwmath}
%\usepackage{mdwtab}
% Also highly recommended is Mark Wooding's extremely powerful MDW tools,
% especially mdwmath.sty and mdwtab.sty which are used to format equations
% and tables, respectively. The MDWtools set is already installed on most
% LaTeX systems. The lastest version and documentation is available at:
% http://www.ctan.org/tex-archive/macros/latex/contrib/mdwtools/


% IEEEtran contains the IEEEeqnarray family of commands that can be used to
% generate multiline equations as well as matrices, tables, etc., of high
% quality.


%\usepackage{eqparbox}
% Also of notable interest is Scott Pakin's eqparbox package for creating
% (automatically sized) equal width boxes - aka "natural width parboxes".
% Available at:
% http://www.ctan.org/tex-archive/macros/latex/contrib/eqparbox/





% *** SUBFIGURE PACKAGES ***
%\usepackage[tight,footnotesize]{subfigure}
% subfigure.sty was written by Steven Douglas Cochran. This package makes it
% easy to put subfigures in your figures. e.g., "Figure 1a and 1b". For IEEE
% work, it is a good idea to load it with the tight package option to reduce
% the amount of white space around the subfigures. subfigure.sty is already
% installed on most LaTeX systems. The latest version and documentation can
% be obtained at:
% http://www.ctan.org/tex-archive/obsolete/macros/latex/contrib/subfigure/
% subfigure.sty has been superceeded by subfig.sty.



%\usepackage[caption=false]{caption}
%\usepackage[font=footnotesize]{subfig}
% subfig.sty, also written by Steven Douglas Cochran, is the modern
% replacement for subfigure.sty. However, subfig.sty requires and
% automatically loads Axel Sommerfeldt's caption.sty which will override
% IEEEtran.cls handling of captions and this will result in nonIEEE style
% figure/table captions. To prevent this problem, be sure and preload
% caption.sty with its "caption=false" package option. This is will preserve
% IEEEtran.cls handing of captions. Version 1.3 (2005/06/28) and later 
% (recommended due to many improvements over 1.2) of subfig.sty supports
% the caption=false option directly:
%\usepackage[caption=false,font=footnotesize]{subfig}
%
% The latest version and documentation can be obtained at:
% http://www.ctan.org/tex-archive/macros/latex/contrib/subfig/
% The latest version and documentation of caption.sty can be obtained at:
% http://www.ctan.org/tex-archive/macros/latex/contrib/caption/




% *** FLOAT PACKAGES ***
%
%\usepackage{fixltx2e}
% fixltx2e, the successor to the earlier fix2col.sty, was written by
% Frank Mittelbach and David Carlisle. This package corrects a few problems
% in the LaTeX2e kernel, the most notable of which is that in current
% LaTeX2e releases, the ordering of single and double column floats is not
% guaranteed to be preserved. Thus, an unpatched LaTeX2e can allow a
% single column figure to be placed prior to an earlier double column
% figure. The latest version and documentation can be found at:
% http://www.ctan.org/tex-archive/macros/latex/base/



%\usepackage{stfloats}
% stfloats.sty was written by Sigitas Tolusis. This package gives LaTeX2e
% the ability to do double column floats at the bottom of the page as well
% as the top. (e.g., "\begin{figure*}[!b]" is not normally possible in
% LaTeX2e). It also provides a command:
%\fnbelowfloat
% to enable the placement of footnotes below bottom floats (the standard
% LaTeX2e kernel puts them above bottom floats). This is an invasive package
% which rewrites many portions of the LaTeX2e float routines. It may not work
% with other packages that modify the LaTeX2e float routines. The latest
% version and documentation can be obtained at:
% http://www.ctan.org/tex-archive/macros/latex/contrib/sttools/
% Documentation is contained in the stfloats.sty comments as well as in the
% presfull.pdf file. Do not use the stfloats baselinefloat ability as IEEE
% does not allow \baselineskip to stretch. Authors submitting work to the
% IEEE should note that IEEE rarely uses double column equations and
% that authors should try to avoid such use. Do not be tempted to use the
% cuted.sty or midfloat.sty packages (also by Sigitas Tolusis) as IEEE does
% not format its papers in such ways.


%\ifCLASSOPTIONcaptionsoff
%  \usepackage[nomarkers]{endfloat}
% \let\MYoriglatexcaption\caption
% \renewcommand{\caption}[2][\relax]{\MYoriglatexcaption[#2]{#2}}
%\fi
% endfloat.sty was written by James Darrell McCauley and Jeff Goldberg.
% This package may be useful when used in conjunction with IEEEtran.cls'
% captionsoff option. Some IEEE journals/societies require that submissions
% have lists of figures/tables at the end of the paper and that
% figures/tables without any captions are placed on a page by themselves at
% the end of the document. If needed, the draftcls IEEEtran class option or
% \CLASSINPUTbaselinestretch interface can be used to increase the line
% spacing as well. Be sure and use the nomarkers option of endfloat to
% prevent endfloat from "marking" where the figures would have been placed
% in the text. The two hack lines of code above are a slight modification of
% that suggested by in the endfloat docs (section 8.3.1) to ensure that
% the full captions always appear in the list of figures/tables - even if
% the user used the short optional argument of \caption[]{}.
% IEEE papers do not typically make use of \caption[]'s optional argument,
% so this should not be an issue. A similar trick can be used to disable
% captions of packages such as subfig.sty that lack options to turn off
% the subcaptions:
% For subfig.sty:
% \let\MYorigsubfloat\subfloat
% \renewcommand{\subfloat}[2][\relax]{\MYorigsubfloat[]{#2}}
% For subfigure.sty:
% \let\MYorigsubfigure\subfigure
% \renewcommand{\subfigure}[2][\relax]{\MYorigsubfigure[]{#2}}
% However, the above trick will not work if both optional arguments of
% the \subfloat/subfig command are used. Furthermore, there needs to be a
% description of each subfigure *somewhere* and endfloat does not add
% subfigure captions to its list of figures. Thus, the best approach is to
% avoid the use of subfigure captions (many IEEE journals avoid them anyway)
% and instead reference/explain all the subfigures within the main caption.
% The latest version of endfloat.sty and its documentation can obtained at:
% http://www.ctan.org/tex-archive/macros/latex/contrib/endfloat/
%
% The IEEEtran \ifCLASSOPTIONcaptionsoff conditional can also be used
% later in the document, say, to conditionally put the References on a 
% page by themselves.





% *** PDF, URL AND HYPERLINK PACKAGES ***
%
%\usepackage{url}
% url.sty was written by Donald Arseneau. It provides better support for
% handling and breaking URLs. url.sty is already installed on most LaTeX
% systems. The latest version can be obtained at:
% http://www.ctan.org/tex-archive/macros/latex/contrib/misc/
% Read the url.sty source comments for usage information. Basically,
% \url{my_url_here}.





% *** Do not adjust lengths that control margins, column widths, etc. ***
% *** Do not use packages that alter fonts (such as pslatex).         ***
% There should be no need to do such things with IEEEtran.cls V1.6 and later.
% (Unless specifically asked to do so by the journal or conference you plan
% to submit to, of course. )


% correct bad hyphenation here
\hyphenation{op-tical net-works semi-conduc-tor}


\begin{document}
%
% paper title
% can use linebreaks \\ within to get better formatting as desired
\title{Conductance of quantum point contact}
%
%
% author names and IEEE memberships
% note positions of commas and nonbreaking spaces ( ~ ) LaTeX will not break
% a structure at a ~ so this keeps an author's name from being broken across
% two lines.
% use \thanks{} to gain access to the first footnote area
% a separate \thanks must be used for each paragraph as LaTeX2e's \thanks
% was not built to handle multiple paragraphs
%

\author{{Guanglei Wang,
        Dragica Vasileska
        }% <-this % stops a space
\thanks{G.-L. Wang is a graduate student of the School of Electrical, Computer, and Energy Engineering, Arizona State
University, Tempe, AZ, 85287 USA e-mail: glwang@asu.edu.}% <-this % stops a space
\thanks{D. Vasileska is a professor of the School of Electrical, Computer, and Energy Engineering, Arizona State
University, Tempe, AZ, 85287 USA e-mail: vasileska@asu.edu.}% <-this % stops a space
}

% The paper headers
\markboth{Final Project of EEE532, Spring 2015}%
{Wang \MakeLowercase{\textit{et al.}}: Conductance of quantum point contact}
% make the title area
\maketitle


\begin{abstract}
    The classic Landauer's formula relates the conductance of a quantum point contact or wire to transmission
    probabilities. There are mainly two approaches to calculate the conductance based on this formula: the Green's
    function techniques and the mode matching techniques. The Green's function approach is frequently used to calculate
    the total transmission probabilities while the mode matching approach, which is based on the scattering
    wavefunction, is normally exploited to calculate the partial transmission probabilities. In this final project of
    ASU EEE532, Spring 2015, we try to learn the mode matching approach formulated by Ando [Phys.Rev. B {\bf 44} , 8017
    (1991)] \cite{A:1991}. We calculate the conductance of a quantum point contact under a confinement potential as well
    as the gate voltage. We also consider the effects of short-range disorders. The results show clear quantum
    quantization phenomenon for the quantum point contact and we discuss several issues that affect the quantized
    behavior.
\end{abstract}
% IEEEtran.cls defaults to using nonbold math in the Abstract.
% This preserves the distinction between vectors and scalars. However,
% if the journal you are submitting to favors bold math in the abstract,
% then you can use LaTeX's standard command \boldmath at the very start
% of the abstract to achieve this. Many IEEE journals frown on math
% in the abstract anyway.

% Note that keywords are not normally used for peerreview papers.
\begin{IEEEkeywords}
Transfer matrix, Green's function, quantum point contact.
\end{IEEEkeywords}

% For peer review papers, you can put extra information on the cover page as needed:
% \ifCLASSOPTIONpeerreview
% \begin{center} \bfseries EDICS Category: 3-BBND \end{center}
% \fi
%
% For peerreview papers, this IEEEtran command inserts a page break and
% creates the second title. It will be ignored for other modes.
\IEEEpeerreviewmaketitle

\section{Introduction}
The developments of fabrication technology have made it possible to obtain quantum ballistic structure of nanoscale,
where the quantum effects can be dominated. Because the small scale of the device, its transport behavior should be
understood based on a atom level theory. Landauer-Buttiker approach is such a classic theory to relate the conductance
with the transmission probabilities within the linear response regime. Normally, both the Green's function and mode
matching technicals are used to calculate the conductance. And the two technicals have been proved to be equivalent. In
Green's function technical, a small imaginary part must be added to the energy to distinguish between the retarded and
advanced Green's function while the energy used in the mode matching technical is totally real. The mode matching
technical is formulated by Ando, which is based on matching the wavefunction in the scattering regime to the Bloch modes
of the lead. In this way we can easily calculate the transmission of each mode. Considering the advantages of the mode
matching technical and its widely using, in this final project, we try to learn the basic knowledge of the mode matching
technical and use it to calculate the quantum conductance of a quantum point contact.

This report is organized as follows. In Sec.\ref{Model}, we briefly introduce the form of Hamiltonian in the lattice
model under the tight-binding approximation. We consider the situation of the ideal wire, the scattering wire and talk
about detail approaches of calculations. In Sec.\ref{Results}, we show our results of the behavior of the conductance
under the variation of some system parameters. The main conclusions are made in Sec.\ref{Conclusion}.

\section{Model}\label{Model}
We consider a square lattice with lattice constant $a$ under the tight-binding approximation. Each atom has a pair of
coordinates $(i,j)$ to represent its position. The nearest-neighbor interaction is characterized by the transfer
integral, $-t$. In this report we use $t=E_F(\frac{1}{2\pi})^2(\frac{\lambda_F}{a})^2$. If the system is under the
effect of magnetic field with Landau gauge, the interaction Hamiltonian of the system can be written as
\begin{equation}
    \begin{split}
        \langle l,j|\mathcal{H}|l+1,j\rangle&=-texp(2\pi i\tilde{H}j) \\
        \langle l,j|\mathcal{H}|l,j+1\rangle&=-t,
    \end{split}
    \label{eq:H}
\end{equation}
where there is a so-called Peierls phase factor $exp(2\pi i\tilde{H}j), j=1,\dots,M$. Here the strength of magnetic
field is introduced through $\tilde{H}=\Phi/\Phi_0$ with $\Phi=Ha^2$ being the magnetic flux across a unit cell and
$\Phi_0=ch/e$ is the magnetic flux quantum. The magnetic field is expressed as
$\tilde{H}=\pi\frac{\hbar\omega_c}{E_F}(\frac{a}{\lambda_F})^2$. The on-site energy is expressed as the diagonal element
of the Hamiltonian which is written as
\begin{equation}
    \langle l,j|\mathcal{H}|l,j\rangle=4t+v(l,j),
    \label{eq:H2}
\end{equation}
where the first term on the right hand side is nothing but a shift of the total energy and the second term denotes the
confinement energy. To consider the effect of short-range disorders, we can just add an random potential to each
diagonal element of the Hamiltonian. The strength of the random disorder is within $[-W/2,W/2]$. Here we have
\begin{equation}
    \frac{W}{E_F}=(\frac{6\lambda_F^3}{\pi^3a^2\Lambda})^{1/2},
    \label{}
\end{equation}
where $\Lambda$ is the mean free path of the electrons.

\subsection{Ideal wire}
The simplest situation we can have in the quantum transport problem is an ideal wire, i.e., $v(l,j)=0$ for all the
lattice points. Suppose the ideal wire is infinite along the $x$ direction but is consisted of $M$ lattice sites in the
$y$ direction. As we are under the tight-binding approximation, the equation of motion can be described as
\begin{equation}
    (E-\mathcal{H}_0)\psi_i+tP\psi_{i-1}+tP^*\psi_{i+1}=0,
    \label{eq:motion}
\end{equation}
where $\psi_i$ is a vector of the amplitudes of the $i$th cell and $P$ is the coupling Hamiltonian between the nearest
cells:
\begin{equation}
    P_{ll'}=exp(2\pi i\tilde{H}l)\delta_{l,l'},
    \label{eq:P}
\end{equation}
where $l,l'=1,\dots,M$. 

As in the ideal wire, the Hamiltonian for each single slice is the same, i.e., $\mathcal{H}_0$, which is also an
$M\times M$ matrix:
\begin{equation}
    \mathcal{H}_0=\left[
        \begin{array}{ccccc}
            v_1+4t & -t & 0 & \dots & 0 \\
            -t & v_2+4t & -t & \dots & 0 \\
            0 & -t & v_3+4t & \dots & 0 \\
            \vdots & \vdots & \vdots & \ddots & \vdots \\
            0 & 0 & 0 & \dots & v_M+4t
        \end{array} \right].
    \label{eq:H0}
\end{equation}

The advantage of the ideal wire is that the Bloch theorem can be easily used, i.e., the wavefunction should be
constructed by a function which has the same period as the potential multiplies an envelop modulation. So we can assume
\begin{equation} \psi_i=\lambda^i\psi_0.  \label{eq:bloch} \end{equation}

Then from the equation of motion we can have
\begin{equation*}
    \lambda\psi_i=t^{-1}P(\mathcal{H}_0-E)\psi_i-P^2\psi_{i-1},
\end{equation*}
which can lead to a $2M\times2M$ eigenvalue problem:
\begin{equation}
    \lambda\left[
        \begin{array}{c}
            \psi_i \\ \psi_{i-1}
        \end{array}
    \right]=\left[
        \begin{array}{cc}
            t^{-1}P(\mathcal{H}_0-E) & -P^2 \\ I & 0
    \end{array}\right]
    \left[ \begin{array}{c}
            \psi_i \\ \psi_{i-1}
    \end{array}\right].
    \label{eq:eig}
\end{equation}

Here in the above equation, $E$ is understood as the chemical potential which can be setup through the gate voltage. For
a given gate energy $E$, we can have an eigenvalue problem which can result in $2M$ eigenvalues and $2M$ eigenvectors.
The dimension of the eigenvector is $2M\times1$ which comes from the fact that the above equation couples $\psi_i$ and
$\psi_{i+1}$, and each of them is of dimension $M\times1$. Based on the form of our equation, the first $M$ elements
of the eigenvector belong to $\psi_i$ and the following $M$ elements belong to $\psi_{i-1}$. The origin of the
$2M$ eigenvalues is that we should have $M$ left-going and $M$ right-going waves. Of course, there should be propagating
modes and evanescent modes. One can distinguish right- and left-going evanescent modes based on the magnitudes of these
corresponding eigenvalues, i.e., right-going evanescent modes have $|\lambda(+)|<1$ and left-going evanescent modes have
$|\lambda(-)|>1$. However, one can not tell the direction of the propagating modes directly through the eigenvalues or
wave vectors. Actually, the quantity that is concerned is the group velocity. Propagating modes have unit eigenvalue
magnitude but different signs of group velocities \cite{KBKZK:2005,BS:1989}. The group velocity for a given mode is
expressed as \cite{KBKZK:2005}
\begin{equation} v_n(\pm)=-\frac{2a}{\hbar}Im[\lambda_n(\pm){\bf
    u}_n(\pm)^\dagger\mathcal{H}_{i,i+1}^\dagger{\bf u}_n(\pm)],
    \label{}
\end{equation}
so the right propagating mode has positive group velocity while the left propagating mode has a negative one.

After distinguishing the left- and right-going modes we can construct some useful matrices, which are
\begin{equation}
    U(\pm)=({\bf u}_1(\pm),\dots,{\bf u}_M(\pm))
    \label{}
\end{equation}
and
\begin{equation}
    \Lambda(\pm)=\left[
        \begin{array}{ccc}
            \lambda_1(\pm) & \dots & 0 \\
            \vdots & \ddots & \vdots \\
            0 & \dots & \lambda_M(\pm)
    \end{array}\right].
    \label{}
\end{equation}
Here the matrix $U(\pm)$ is nothing but a group of eigenvectors for the right(left) going modes. We will later see that it
is used to calculate how much transmission each mode contributes. At last we need to define
\begin{equation}
    F(\pm)=U(\pm)\Lambda(\pm)U^{-1}(\pm).
    \label{}
\end{equation}

\subsection{Scattering problem}
After obtaining the transport modes for the ideal wire, we now move on to the scattering problem. The scattering can be
as simple as the effect of the confinement potential. When the potential profile of the device is not uniform, the
states of electrons will be disturbed, i.e., scattering will happen. The basic approach to deal with this scattering in
the mode matching technical \cite{BKJ:1991} is to calculate the relation of the out-going waves with the in-going waves.
For example, if we concern transmission, we should focus on the right propagating waves on the most left and right parts
of device. If we consider the reflection, we should focus on the left propagating waves on the most left part of the
device. Normally, we assume the leads are semi-infinite and ideal and the length of the scattering regime is $N$, i.e.,
is consisted of $N$ lattice points.

To be concrete, we consider the interface of the left lead and the device
\begin{equation}
    \psi_0=\psi_0(+)+\psi_0(-).
    \label{}
\end{equation}
And we can also relate $\psi_{-1}$ to $\psi_{0}$ through the $F(\pm)$ matrices
\begin{equation}
    C_{-1}=F^{-1}(+)\psi_0(+)+F^{-1}(-)\psi_0(-).
    \label{}
\end{equation}
Then we can exploit the equation of motion at cell $0$ to get
\begin{equation}
    (E-\tilde{\mathcal{H}}_0)\psi_0=tP^*\psi_1=-tP[F^{-1}(+)-F^{-1}(-)]\psi_0(+),
    \label{eq:h0}
\end{equation}
where
\begin{equation}
    \tilde{\mathcal{H}}_0=\mathcal{H}_0-tPF^{-1}(-).
    \label{}
\end{equation}

Similarly, for the most right part of the device we can have
\begin{equation}
    (E-\tilde{\mathcal{H}}_{N+1})\psi_{N+1}+tP\psi_N=0,
    \label{}
\end{equation}
where 
\begin{equation}
    \tilde{\mathcal{H}}_{N+1}=\mathcal{H}_{N+1}-tP^*F(+).
    \label{}
\end{equation}
Here $\tilde{\mathcal{H}}_0$ and $\tilde{\mathcal{H}}_{N+1}$ actually already contain the effects coming from the two
semi-infinite leads. We should notice that the right hand side of Eq.\ref{eq:h0} is not $0$. Indeed, the term
$-tP[F^{-1}(+)-F^{-1}(-)]\psi_0(+)$ serves as a source of the current.

To calculate the effect of the source term on the most right part of the device, we can use the Green's function
technical. Here, we should emphasis that the Green's function used here is just a tool to calculate the wavefunction on
the most right side of the device, after which we should use mode matching technicals to match the wavefunction to the
eigenfunctions of the ideal leads. We can define the Green's function $G$ as
\begin{equation}
    G=\frac{1}{E-\tilde{\mathcal{H}}},
    \label{}
\end{equation}
with
\begin{equation}
    \tilde{\mathcal{H}}=\left[
        \begin{array}{cccccc}
            \tilde{\mathcal{H}}_0 & -tP^* & 0 & \dots & 0 & 0 \\
            -tP & \tilde{\mathcal{H}}_1 & -tP^* & \dots & 0 & 0 \\
            0 & -tP & \tilde{\mathcal{H}}_2 & \dots & 0 & 0 \\
            \vdots & \vdots & \vdots & \ddots & \vdots & \vdots \\
            0 & 0 & 0 & \dots & \tilde{\mathcal{H}}_N & -tP^* \\
            0 & 0 & 0 & \dots & -tP & \tilde{\mathcal{H}}_{N+1}
    \end{array}\right].
    \label{}
\end{equation}

The role of the source is reflected in the coupling of $\psi_0$ and $\psi_{N+1}$
\begin{equation}
    \begin{split}
    \psi_{N+1}(+)&=\psi_{N+1} \\
    &=-t\langle N+1|G|0\rangle P[F^{-1}(+)-F^{-1}(-)]\psi_0(+).
\end{split}
    \label{}
\end{equation}

From which the transmission coefficient $t_{\mu\nu}$ for the incident channel $\nu$ to the out-going channel $\mu$ is
\begin{equation}
    \begin{split}
    t_{\mu\nu}=&(\frac{v_\mu}{v_\nu})^{1/2}\{-tU^{-1}(+)\langle N+1|G|0\rangle \\
    &\times P[F^{-1}(+)-F^{-1}(-)]U(+)\}_{\mu\nu},
\end{split}
    \label{}
\end{equation}
where $v_\nu$ and $v_\mu$ are the velocity for each channel, respectively.

\subsection{Recursive formulas}
As there are matrix inverse manipulations in the calculations of Green's function, we can exploit an recursive formulas
to speed up the calculation. The essence of this formulas is to replace the inverse of a very large matrix (the same
dimension as the number of atoms in the device) with many inverses of small matrices (the same dimension as the number
of atoms in the slice) \cite{A:1990}. To be concrete, in our device , the recursive formulas is
\begin{equation}
    \begin{split}
        \langle i+1|G^{(i+1)}|i+&1\rangle^{-1} \\
        &=E-\tilde{\mathcal{H}}_{i+1}-\tilde{\mathcal{H}}_{i+1,i}\langle i|G^{(i)}|i\rangle\tilde{\mathcal{H}}_{i,i+1},
        \\
        \langle i+1|G^{(i+1)}|0\rangle& \\
        &=\langle i+1|G^{(i+1)}|i+1\rangle\tilde{\mathcal{H}}_{i+1,i}\langle i|G^{(i)}|0\rangle,
    \end{split}
    \label{}
\end{equation}
the initial condition is $\langle0|G^{(0)}|0\rangle=(E-\tilde{\mathcal{H}}_0)^{-1}$.

\section{Results}\label{Results}

\begin{figure}
    \centering
    \epsfig{figure=figure1.eps,width=\linewidth}
    \caption{An example of the confinement potential profile used to simulate a quantum point contact. $(a)$ Confinement
    potential along $y$ direction under several different $x$ positions. $(b)$ A contour view of the confinement
potential of the quantum point contact. Note that the scale for the $x$ and $y$ axises are different. The other
parameters are $V/E_F=0.5$, $\Delta/\lambda_F=1$, and $L_y/\lambda_F=4$. The solid lines in $(a)$ represents the
potential profile at $x/L_x=0, 0.66, 0.75, 0.83$.}
    \label{fig1}
\end{figure}

We consider a wire with length $L_x$ and width $L_y$ and under the influence of a confinement potential
\begin{equation}
    \begin{split}
        V(x,y)=&\frac{V}{2}[1+cos(\frac{2\pi x}{L_x})] \\
        &+E_F(\frac{y-y_+(x)}{\Delta})^2\theta(y-y_+(x)) \\
        &+E_F(\frac{y-y_-(x)}{\Delta})^2\theta(-(y-y_-(x))),
    \end{split}
    \label{}
\end{equation}
where $\theta(t)$ is a step function and 
\begin{equation}
    y_{\pm}(x)=\pm\frac{L_y}{4}[1-cos(\frac{2\pi x}{L_x})].
    \label{}
\end{equation}
In Fig.\ref{fig1} we show a typical profile of this confinement potential. Fig.\ref{fig1} $(a)$ is the example of the
confinement potential under some different $x$ positions. The minimum energy under $x/L_x=0$ is called the bottom
energy.  Fig.\ref{fig1} $(b)$ is a contour figure of the confinement potential all over the device. We use a much
smaller mesh than that used in simulation to show the detail of this confinement. 

\begin{figure} \centering
    \epsfig{figure=figure2.eps,width=\linewidth} \caption{Total conductance through a quantum point contact under
        different bottom energy $V/E_F$. The lines with different colors represent different gate voltage, i.e., red:
        $E/E_F=1.0$, blue: $E/E_F=2.0$, pink: $E/E_F=0.5$.  The different line styles represents different length of the
    quantum point contact, i.e., solid: $L_x/\lambda_F=16.0$, dash: $L_x/\lambda_F=8.0$, dash-dot: $L_x/\lambda_F=4.0$}
    \label{fig2}
\end{figure}

Fig.\ref{fig2} shows the dependence of the total conductance of the quantum point contact on the bottom energy,
$V/E_F$. The red, blue and pink curves are correspondening to the gate energy $E/E_F=2.0, 1.0, 0.5$, respectively. For
the different line styles under the same color, they represent different lengths of the device. The blue lines are the
same as that in the work of T. Ando's, which proves the valid of our program. We can see that for a rather long device,
there is a very clear conductance quantization phenomenon. We can see the conductance steps obviously. However, when the
length of the device becomes smaller, the conductance becomes less quantized as the potential variation is far from
being adiabatic. Furthermore, we can also see that the gate voltage also plays a significant role here. When the gate
energy, or the chemical potential, is high, there should be more propagating modes in the leads, which increases the
conductance. Another physical picture to understand this is through the band structure. As we increase the gate
voltage, the chemical potential will have more crosses with the band structure, and more crosses mean more transporting
modes.  In Fig.\ref{fig2}, we can also see the effect of the bottom energy very well. We can see that for a long device,
a bottom energy which is lower than the gate energy is high enough to block the transport of the device. However, for a
short device, it is reasonable that the transport should be robust than that of the long one.

\begin{figure}
    \centering
    \epsfig{figure=figure3.eps,width=\linewidth}
    \caption{Conductance distribution among out-going channels as a function of the bottom energy $V/E_F$ through the
    quantum point contact. The blue curves are conductances for different single channels and the red curves are for sum
channels. $C_i$ denotes channel $i$ with $i=1,2,3,4$. The gate energy used here is $E/E_F=1.0$ so that there are four
propagating modes in the ideal lead.}
    \label{fig3}
\end{figure}

To exploit the advantage of the mode matching approach, we show the quantum conductance for each mode separately in
Fig.\ref{fig3}. It shows the detail of the variation of each propagating mode with the change of the bottom energy. It
clearly shows that the conductance can not go beyond $e^2/\pi\hbar$ for one channel, which results from the Fermi
property, or the Pauli exclusion principle, of electrons. We can see that when the bottom energy becomes higher and the
total conductance jumps to a lower value, the conductance of the highest channel jumps from close to unity to close one.
This is because the confinement potential blocks the transport of that channel. Another interesting phenomenon is that
the conductance of the very high channel's can keep finite, even close to zero, within a rather large bottom energy
regime.
\begin{figure}
    \centering
    \epsfig{figure=figure4.eps,width=\linewidth}
    \caption{Total conductance through a quantum point contact under different gate energy $E/E_F$. The lines with
    different colors represent different bottom energy, i.e., red: $V/E_F=1.0$, blue: $V/E_F=0.5$.  The different line
styles represents different length of the quantum point contact, i.e., solid: $L_x/\lambda_F=16.0$, dash:
$L_x/\lambda_F=8.0$, dash-dot: $L_x/\lambda_F=4.0$} \label{fig4}
\end{figure}
\begin{figure}
    \centering
    \epsfig{figure=figure5.eps,width=\linewidth}
    \caption{Conductance distribution among out-going channels as a function of the gate energy $E/E_F$ through the
    quantum point contact. The blue curves are conductances for different single channels and the red curves are for sum
channels. $C_i$ denotes channel $i$ with $i=1,2,3,4$. When the gate energy is higher than $E_F$, more propagating modes
can be existed in the quantum point contact. Here we just show the first four channels.}
    \label{fig5}
\end{figure}

Except the dependence of the conductance on the bottom energy, we also show the relation of the conductance with the
gate energy $E/E_F$. The results are shown in Fig.\ref{fig4}. We do two groups of simulations with the bottom energy
$V/E_F=0.5, 1.0$ and also under three different device lengths. From Fig.\ref{fig4}, we can see a typical ribbon
conductance behavior. The conductance starts to appear above the bottom energy of each case, which is shown by the
vertical dot line. Again, the length of the device can change the quantization behavior. The detail variation of each
channel is shown in Fig.\ref{fig5}. An interesting behavior of the channel conductance is that there is a trend to
decrease after it reaches the maximum value, i.e., one, and then increases back to unity. Also, from the behavior of the
single channel, we can inform the minimum energy of several bands in the band structure.

\begin{figure}
    \centering
    \epsfig{figure=figure6.eps,width=\linewidth}
    \caption{Examples of the conductances through a quantum point contact as a function of the bottom energy under the
    effect of short-range disorders characterized by the mean free path $\Lambda$. Here we show 100 realizations of the
random disorders with $\Lambda/\lambda_F=200.0$, blue dots, and $\Lambda/\lambda_F=20.0$, red dots, respectively. The
gate energy we used is $E/E_F=1.0$.}
    \label{fig6}
\end{figure}

\begin{figure}
    \centering
    \epsfig{figure=figure7.eps,width=\linewidth}
    \caption{Examples of the conductances through a quantum point contact as a function of the gate energy under the
    effect of short-range disorders characterized by the mean free path $\Lambda$. Here we show 100 realizations of the
random disorders with $\Lambda/\lambda_F=200.0$, blue dots, and $\Lambda/\lambda_F=20.0$, red dots, respectively. The
bottom energy we used is $V/E_F=0.5$.}
    \label{fig7}
\end{figure}

Another interesting question is the effect of disorders on the conductance of the quantum point contact. Of course,
there are many kinds of disorders, and here we use the simplest one, i.e., the short-range random disorder.
Mathematically, we just need to add a random potential to each diagonal element of the Hamiltonian. The simulation
results are shown in Fig.\ref{fig6} and Fig.\ref{fig7} with the variation of the bottom energy and gate energy,
respectively. For each case, we make 100 realizations of the random disorder with the same strength, $W$. The strength
of the disorder is characterized by the mean free path $\Lambda$. The blue dots are for $\Lambda/\lambda_F=200.0$ and
the red dots are for $\Lambda/\lambda_F=20.0$, where a larger mean free path denotes a weaker disorder. The results are
agreed with the expectation. There is a clear signature that the blue dots are more focused than the red dots. And a
stronger disorder can even destroy the quantization of the conductance. Again, the gate energy we used in Fig.\ref{fig6}
is $E/E_F=1.0$ and the bottom energy we used in Fig.\ref{fig7} is $V/E_F=0.5$.

\section{Conclusion}\label{Conclusion}

We study the transport properties of a quantum point contact under a confinement. We calculate the conductance of the
device with the change of the bottom energy and gate energy. We see that the quantum conductance are quantized for a
relative long device. However, the quantization becomes unclear as we decrease the length of the device. We also study
the partial transport of each channel in the device. The results show that each channel can contribute one unity of
conductance at best. The higher the gate voltage is, the more the transport channels are. At last, we study the effects
of short-range disorders on the transport behavior. As expected, the disorder makes an overall decrease of the
conductance and makes the quantization less obvious.

\section*{Acknowledgment}


G.-L. Wang would like to thank Prof. Richard Akis for helpful discussion. And thank Prof. Dragica Vasileska for her
wonderful lectures in ASU EEE532, Spring 2015.

\bibliographystyle{IEEEtran}
\bibliography{IEEEabrv,Final}

% that's all folks
\end{document}


